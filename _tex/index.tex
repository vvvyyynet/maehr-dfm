% Options for packages loaded elsewhere
\PassOptionsToPackage{unicode}{hyperref}
\PassOptionsToPackage{hyphens}{url}
\PassOptionsToPackage{dvipsnames,svgnames,x11names}{xcolor}
%
\documentclass[
  letterpaper,
  DIV=11,
  numbers=noendperiod]{scrartcl}

\usepackage{amsmath,amssymb}
\usepackage{iftex}
\ifPDFTeX
  \usepackage[T1]{fontenc}
  \usepackage[utf8]{inputenc}
  \usepackage{textcomp} % provide euro and other symbols
\else % if luatex or xetex
  \usepackage{unicode-math}
  \defaultfontfeatures{Scale=MatchLowercase}
  \defaultfontfeatures[\rmfamily]{Ligatures=TeX,Scale=1}
\fi
\usepackage{lmodern}
\ifPDFTeX\else  
    % xetex/luatex font selection
\fi
% Use upquote if available, for straight quotes in verbatim environments
\IfFileExists{upquote.sty}{\usepackage{upquote}}{}
\IfFileExists{microtype.sty}{% use microtype if available
  \usepackage[]{microtype}
  \UseMicrotypeSet[protrusion]{basicmath} % disable protrusion for tt fonts
}{}
\makeatletter
\@ifundefined{KOMAClassName}{% if non-KOMA class
  \IfFileExists{parskip.sty}{%
    \usepackage{parskip}
  }{% else
    \setlength{\parindent}{0pt}
    \setlength{\parskip}{6pt plus 2pt minus 1pt}}
}{% if KOMA class
  \KOMAoptions{parskip=half}}
\makeatother
\usepackage{xcolor}
\setlength{\emergencystretch}{3em} % prevent overfull lines
\setcounter{secnumdepth}{5}
% Make \paragraph and \subparagraph free-standing
\makeatletter
\ifx\paragraph\undefined\else
  \let\oldparagraph\paragraph
  \renewcommand{\paragraph}{
    \@ifstar
      \xxxParagraphStar
      \xxxParagraphNoStar
  }
  \newcommand{\xxxParagraphStar}[1]{\oldparagraph*{#1}\mbox{}}
  \newcommand{\xxxParagraphNoStar}[1]{\oldparagraph{#1}\mbox{}}
\fi
\ifx\subparagraph\undefined\else
  \let\oldsubparagraph\subparagraph
  \renewcommand{\subparagraph}{
    \@ifstar
      \xxxSubParagraphStar
      \xxxSubParagraphNoStar
  }
  \newcommand{\xxxSubParagraphStar}[1]{\oldsubparagraph*{#1}\mbox{}}
  \newcommand{\xxxSubParagraphNoStar}[1]{\oldsubparagraph{#1}\mbox{}}
\fi
\makeatother


\providecommand{\tightlist}{%
  \setlength{\itemsep}{0pt}\setlength{\parskip}{0pt}}\usepackage{longtable,booktabs,array}
\usepackage{calc} % for calculating minipage widths
% Correct order of tables after \paragraph or \subparagraph
\usepackage{etoolbox}
\makeatletter
\patchcmd\longtable{\par}{\if@noskipsec\mbox{}\fi\par}{}{}
\makeatother
% Allow footnotes in longtable head/foot
\IfFileExists{footnotehyper.sty}{\usepackage{footnotehyper}}{\usepackage{footnote}}
\makesavenoteenv{longtable}
\usepackage{graphicx}
\makeatletter
\newsavebox\pandoc@box
\newcommand*\pandocbounded[1]{% scales image to fit in text height/width
  \sbox\pandoc@box{#1}%
  \Gscale@div\@tempa{\textheight}{\dimexpr\ht\pandoc@box+\dp\pandoc@box\relax}%
  \Gscale@div\@tempb{\linewidth}{\wd\pandoc@box}%
  \ifdim\@tempb\p@<\@tempa\p@\let\@tempa\@tempb\fi% select the smaller of both
  \ifdim\@tempa\p@<\p@\scalebox{\@tempa}{\usebox\pandoc@box}%
  \else\usebox{\pandoc@box}%
  \fi%
}
% Set default figure placement to htbp
\def\fps@figure{htbp}
\makeatother
% definitions for citeproc citations
\NewDocumentCommand\citeproctext{}{}
\NewDocumentCommand\citeproc{mm}{%
  \begingroup\def\citeproctext{#2}\cite{#1}\endgroup}
\makeatletter
 % allow citations to break across lines
 \let\@cite@ofmt\@firstofone
 % avoid brackets around text for \cite:
 \def\@biblabel#1{}
 \def\@cite#1#2{{#1\if@tempswa , #2\fi}}
\makeatother
\newlength{\cslhangindent}
\setlength{\cslhangindent}{1.5em}
\newlength{\csllabelwidth}
\setlength{\csllabelwidth}{3em}
\newenvironment{CSLReferences}[2] % #1 hanging-indent, #2 entry-spacing
 {\begin{list}{}{%
  \setlength{\itemindent}{0pt}
  \setlength{\leftmargin}{0pt}
  \setlength{\parsep}{0pt}
  % turn on hanging indent if param 1 is 1
  \ifodd #1
   \setlength{\leftmargin}{\cslhangindent}
   \setlength{\itemindent}{-1\cslhangindent}
  \fi
  % set entry spacing
  \setlength{\itemsep}{#2\baselineskip}}}
 {\end{list}}
\usepackage{calc}
\newcommand{\CSLBlock}[1]{\hfill\break\parbox[t]{\linewidth}{\strut\ignorespaces#1\strut}}
\newcommand{\CSLLeftMargin}[1]{\parbox[t]{\csllabelwidth}{\strut#1\strut}}
\newcommand{\CSLRightInline}[1]{\parbox[t]{\linewidth - \csllabelwidth}{\strut#1\strut}}
\newcommand{\CSLIndent}[1]{\hspace{\cslhangindent}#1}

\KOMAoption{captions}{tableheading}
\makeatletter
\@ifpackageloaded{tcolorbox}{}{\usepackage[skins,breakable]{tcolorbox}}
\@ifpackageloaded{fontawesome5}{}{\usepackage{fontawesome5}}
\definecolor{quarto-callout-color}{HTML}{909090}
\definecolor{quarto-callout-note-color}{HTML}{0758E5}
\definecolor{quarto-callout-important-color}{HTML}{CC1914}
\definecolor{quarto-callout-warning-color}{HTML}{EB9113}
\definecolor{quarto-callout-tip-color}{HTML}{00A047}
\definecolor{quarto-callout-caution-color}{HTML}{FC5300}
\definecolor{quarto-callout-color-frame}{HTML}{acacac}
\definecolor{quarto-callout-note-color-frame}{HTML}{4582ec}
\definecolor{quarto-callout-important-color-frame}{HTML}{d9534f}
\definecolor{quarto-callout-warning-color-frame}{HTML}{f0ad4e}
\definecolor{quarto-callout-tip-color-frame}{HTML}{02b875}
\definecolor{quarto-callout-caution-color-frame}{HTML}{fd7e14}
\makeatother
\makeatletter
\@ifpackageloaded{caption}{}{\usepackage{caption}}
\AtBeginDocument{%
\ifdefined\contentsname
  \renewcommand*\contentsname{Inhaltsverzeichnis}
\else
  \newcommand\contentsname{Inhaltsverzeichnis}
\fi
\ifdefined\listfigurename
  \renewcommand*\listfigurename{Abbildungsverzeichnis}
\else
  \newcommand\listfigurename{Abbildungsverzeichnis}
\fi
\ifdefined\listtablename
  \renewcommand*\listtablename{Tabellenverzeichnis}
\else
  \newcommand\listtablename{Tabellenverzeichnis}
\fi
\ifdefined\figurename
  \renewcommand*\figurename{Abbildung}
\else
  \newcommand\figurename{Abbildung}
\fi
\ifdefined\tablename
  \renewcommand*\tablename{Tabelle}
\else
  \newcommand\tablename{Tabelle}
\fi
}
\@ifpackageloaded{float}{}{\usepackage{float}}
\floatstyle{ruled}
\@ifundefined{c@chapter}{\newfloat{codelisting}{h}{lop}}{\newfloat{codelisting}{h}{lop}[chapter]}
\floatname{codelisting}{Listing}
\newcommand*\listoflistings{\listof{codelisting}{Listingverzeichnis}}
\makeatother
\makeatletter
\makeatother
\makeatletter
\@ifpackageloaded{caption}{}{\usepackage{caption}}
\@ifpackageloaded{subcaption}{}{\usepackage{subcaption}}
\makeatother

\ifLuaTeX
\usepackage[bidi=basic]{babel}
\else
\usepackage[bidi=default]{babel}
\fi
\babelprovide[main,import]{nswissgerman}
% get rid of language-specific shorthands (see #6817):
\let\LanguageShortHands\languageshorthands
\def\languageshorthands#1{}
\ifLuaTeX
  \usepackage[german]{selnolig} % disable illegal ligatures
\fi
\usepackage{bookmark}

\IfFileExists{xurl.sty}{\usepackage{xurl}}{} % add URL line breaks if available
\urlstyle{same} % disable monospaced font for URLs
\hypersetup{
  pdftitle={Handbuch zur Erstellung diskriminierungssensibler Metadaten für historische Quellen und Forschungsdaten},
  pdflang={de-CH},
  pdfkeywords={Diskriminierungssensible Metadaten, Diskriminierungsfreie
Metadaten, Historische Quellen und
Forschungsdaten, FAIR-Prinzipien, CARE-Prinzipien, Datenethik, Stadt.Geschichte.Basel, Open
Research Data, Code of Conduct, Dublin Core, Schlagwortindex
GenderOpen},
  colorlinks=true,
  linkcolor={blue},
  filecolor={Maroon},
  citecolor={Blue},
  urlcolor={Blue},
  pdfcreator={LaTeX via pandoc}}


\title{Handbuch zur Erstellung diskriminierungssensibler Metadaten für
historische Quellen und Forschungsdaten}
\usepackage{etoolbox}
\makeatletter
\providecommand{\subtitle}[1]{% add subtitle to \maketitle
  \apptocmd{\@title}{\par {\large #1 \par}}{}{}
}
\makeatother
\subtitle{Schlüsselbegriffe und Erfahrungen aus der Praxis}
\author{Levyn Bürki \and Moritz Mähr \and Noëlle Schnegg}
\date{2024-10-31}

\begin{document}
\maketitle
\begin{abstract}
Dieses Handbuch ist ein Leitfaden zur Erstellung von
diskriminierungssensibler Metadaten für historische Quellen und
Forschungsdaten. Es richtet sich an Historiker*innen, Archivar*innen,
Bibliothekar*innen und alle, die sich mit Datenethik und Open Research
Data in den Geschichtswissenschaften beschäftigen. Die Autor*innen Levyn
Bürki, Moritz Mähr und Noëlle Schnegg führen durch die praktischen
Aspekte der Erstellung von Metadaten, basierend auf den FAIR- und
CARE-Prinzipien, um Forschungsdaten auffindbar, zugänglich,
interoperabel und nachnutzbar zu machen. Durch praktische Anleitungen
und illustrierte Fallbeispiele zeigt das Handbuch, wie maschinenlesbare
Metadaten Forschung und Lehre bereichern und die Interpretation
historischer Quellen beeinflussen können. Als öffentlich zugängliches
``Living Document'' ist es auf eine kontinuierliche Weiterentwicklung
durch die Community ausgelegt und verpflichtet sich zu einer inklusiven
und diskriminierungsfreien Darstellung historischer Inhalte. Das
Handbuch ist eine grundlegende Ressource für alle, die sich mit moderner
digitaler Geschichtswissenschaft und Open Research Data beschäftigen
wollen.
\end{abstract}


\begin{tcolorbox}[enhanced jigsaw, colbacktitle=quarto-callout-warning-color!10!white, rightrule=.15mm, coltitle=black, left=2mm, opacitybacktitle=0.6, toptitle=1mm, title=\textcolor{quarto-callout-warning-color}{\faExclamationTriangle}\hspace{0.5em}{Content Note}, colback=white, colframe=quarto-callout-warning-color-frame, opacityback=0, titlerule=0mm, leftrule=.75mm, breakable, bottomtitle=1mm, bottomrule=.15mm, arc=.35mm, toprule=.15mm]

Dieses Dokument enthält Abbildungen von historischen Quellen, die
diskriminierende Sprache, Bilder oder Darstellungen enthalten. Sie sind
Ausdruck von Vorurteilen, Stereotypen oder Gewalt gegen bestimmte
Gruppen in der Vergangenheit.

\end{tcolorbox}

\section{Einleitung}\label{einleitung}

\begin{itemize}
\item
  \textbf{Warum dieses Handbuch?}

  \begin{itemize}
  \tightlist
  \item
    \emph{Aktualität Thema:}
  \item
    \emph{Abgrenzung zu anderen Projekten/Leitfäden/Handbüchern:}
  \item
    \emph{Zielgruppe:} Das Handbuch richtet sich an
    Historiker\emph{innen, Archivar}innen, Bibliothekar*innen und alle,
    die sich mit Open Research Data in den Geschichtswissenschaften
    beschäftigen.
  \end{itemize}
\item
  \textbf{Entstehungskontext}

  \begin{itemize}
  \tightlist
  \item
    Eine erste Auflage dieses Handbuchs entstand im Rahmen des
    historischen Forschungsprojekts \emph{Stadt.Geschichte.Basel} und
    wurde im Juni 2024 von Moritz Mähr und Noëlle Schnegg publiziert.
    Gemeinsam mit Levyn Bürki wurde das Handbuch ab Herbst 2024
    schliesslich neu strukturiert und überarbeitet.

    \begin{itemize}
    \tightlist
    \item
      Einarbeitung des Feedbacks aus der Erstveröffentlichung sowie von
      Diskussionen, welche seither an Workshops und Konferenzen geführt.
    \item
      Titeländerung
    \item
      Eigene Positionalität
    \end{itemize}
  \end{itemize}
\item
  \textbf{Problemdefinitionen}

  \begin{itemize}
  \tightlist
  \item
    \textbf{Diskriminierungsbegriff und Datenethik:}

    \begin{itemize}
    \tightlist
    \item
      Diskriminierung kann viele Formen annehmen und geht über
      diskriminierendes Vokabular hinaus.
    \item
      Darunter Rassismus, Antisemitismus, Sexismus, Ableismus,
      Transphobie und viele andere, und sie kann mehrere
      Diskriminierungsformen gleichzeitig enthalten
      (Intersektionalität).
    \item
      Sie kann explizit oder implizit sein und ist oft tief in den
      Kontext und die Interpretation dieser Ressourcen eingebettet.
    \item
      Implizite oder strukturelle Formen von Diskriminierung finden sich
      in vielen bestehenden Thesauri und Schlagwortverzeichnissen wie
      der Gemeinsamen Normdatei (GND), der am weitesten verbreiteten
      Normdatei im deutschsprachigen Raum.
    \item
      Durch Basisfelder, Klassifizierung nach hegemonialer Praxis,
      \ldots{} Bei einem Textfeld geht es immer auch darum, was es
      \emph{kann}: wie wird es ausgelesen? Interpretiert? Dargestellt?
      Übermittlet? Dokumentiert?
    \item
      Sie stecken aber auch in institutionellen Strukturen/Bias,
      kolonialen Sammlungskonzepten, \ldots{}
    \item
      Schliesslich sind auch Aspekte des Datenschutz und
      Selbstbestimmung ein wesentlicher Teil eines
      diskriminierungssensiblen Umgangs mit Daten und Metadaten. Im
      Kontext indigener Daten existieren dazu bereits die
      CARE-Prinzipien, welche auch für andere Kontexte, etwa den Umgang
      mit Daten aus nationalsozialistischen Kontexten wegweisend sein
      können. Datenethische Grundsätze (Do-no-harm Prinzip,
      Schadensminimierung, Collectiv benefit).
    \item
      Unter diskriminierungssensiblem Arbeiten an und mit Archiv- und
      Forschungsdaten verstehen wir unmittelbar auch eine Arbeit an
      Forschungsdatenqualität.
    \end{itemize}
  \item
    \textbf{Beispiele, Fragen und Konfliktlinien:} Inhaltlich,
    strukturell, technisch und politisch:

    \begin{itemize}
    \tightlist
    \item
      Wie können Metadaten transparent gestaltet (etwa hinsichtlich
      Versionierung und Autor:innenschaft) und wie können
      Entscheidungsprozesse dokumentiert werden?
    \item
      Wie soll mit Zielkonflikten (etwa mit Förderrichtlinien)
      umgegangen werden?
    \item
      Wie gelingt diskriminierungssensibler Umgang mit Metadaten mit
      Blick auf limitierte Ressourcen?
    \item
      Umgang mit KI im Kontext der Metadatenerstellung und -verarbeitung
    \item
      DMPs
    \item
      Wie können Arbeitsprozesse diskriminierungssensibel gestaltet
      werden -- von der Sensibilisierung, Schulung und Diversifizierung
      von Teams, der Entwicklung von Tools, \ldots{}
    \item
      Wie kann diskriminierungssensible Arbeit in Forschung, Sammlungen
      und Archiven vor politischer Einflussnahme geschützt und
      verteidigt werden?
    \item
      Was muss bei der Veröffentlichung und Vermittlung beachtet werden
      um diskriminierender Strukturen nicht zu reproduzieren?
    \item
      Kontrollverlust der Darstellung beim Publizieren der Daten auf
      externen Plattformen und Repositorien
    \item
      Wie kann diskriminierendes Vokabular überwunden werden ohne
      historische Quellen zu verfälschen und auch künftig historische
      Forschung nach bestimmten Begriffen zu ermöglichen?
    \item
      Umgang mit Lücken, Unsicherheiten und Widersprüchen in Daten,
      insbesondere bei Massenerschliessungen, wo es auch darum gehen
      muss, systematische Fehler zu limitieren.
    \item
      \ldots{}
    \end{itemize}
  \end{itemize}
\item
  \textbf{Aufbau, Fokus und Verwendungshinweise:} Was das Handbuch
  \emph{ist} -- und was es \emph{nicht ist}. Wie verwende ich das
  Handbuch? Was kann ich erwarten? Ist dieses Handbuch für mich?

  \begin{itemize}
  \tightlist
  \item
    \textbf{Ziel:} Ziel muss es sein, diskriminierende Strukturen nicht
    nur aufzeigen, sondern diese zu überwinden. Dazu braucht es einen
    Austausch über Lösungsansätze genauso wie über bestehende Hürden und
    Unsicherheiten. Hierzu möchte das vorliegende Handbuch einen Beitrag
    leisten. Dazu sammelt es Erfahrungswissen zum
    diskriminierungssensiblen Umgang mit Metadaten in
    Forschungskontexten und gibt diese weiter. Um möglichst vielen
    Menschen zugänglich zu sein, werden in einem ersten Teil wichtige
    Schlüsselbegriffe eingeführt. Ein zweiter Teil versammelt
    schliesslich Praxiswissen aus verschiedenen thematischen,
    strukturellen und institutionellen Kontexten.
  \item
    \textbf{Metadaten-Lebenszyklus:} Erstellung / Verwendung /
    Vermittlung usw.
  \item
    \textbf{Diskriminierungs-Kontexte:} Essenzialisierung, Bias, Lücken
    und Unsichtbarmachung, Fehlende Perspektiven durch
    Unwissen/Ignorance, ethische Konflikte bei sensiblen Daten,
    institutionalisierte Strukturen im Arbeitsalltag,\ldots{}
  \item
    \textbf{Umgang:} Wünschenswert sind auch Beiträge, wie
    diskriminierende Metadaten nicht nur erkannt, sondern auch
    angesprochen und beseitigt/vermieden/überwunden werden können --
    nicht zuletzt im Kontext von Förderrichtlinien oder Teaminternen
    Spannungen.
  \item
    \textbf{Spezifität und Pragmatismus:} Die im Handbuch versammelten
    Empfehlungen müssen freilich auf den Kontext spezifischer
    Institutions-, Sammlungs- und Forschungskontexte angepasst werden.
    Durch einen modularen Aufbau mit verschiednene Perspektiven und
    Beispielen versucht das Handbuch den heterogenen Ansprüchen und
    Verwendungskontexten Rechnung zu tragen.
  \item
    \textbf{Boxen mit Erfahrungen und Fallbeispielen:} Der modulare
    Aufbau soll es ermöglichen, verschiedene Stimmen aus dem Forschungs-
    und Sammlungskontext, sowie aus Public History Projekten zu
    versammeln und Erfahrungen und Lösungsansätze zu teilen.
  \item
    \textbf{Living Document:} Das Handbuch ist als kostenloses und
    öffentlich zugängliches Living Document konzipiert und soll von der
    Community auf dem öffentlichen Code-Repositorium weiterentwickelt
    werden.
  \item
    \textbf{Code of Conduct:}
    \href{https://www.contributor-covenant.org/version/2/1/code_of_conduct/}{Contributor
    Covenant Code of Conduct}
  \end{itemize}
\end{itemize}

\textbf{Dank:} - Ggf. hier Personen verdanken, die Inputs und Beispiele
geliefert und gegengelesen haben.

\section{Schlüsselbegriffe}\label{schluxfcsselbegriffe}

\subsection{Diskriminierung in Daten und
Datensätzen}\label{diskriminierung-in-daten-und-datensuxe4tzen}

\subsection{Forschungsdaten}\label{forschungsdaten}

\subsection{Metadaten}\label{metadaten}

\subsubsection{Was sind Metadaten?}\label{was-sind-metadaten}

\begin{itemize}
\tightlist
\item
  Beispiele und Abgrenzung zu \emph{Daten} und \emph{Paradaten}.
\item
  Was ist ein Metadatenobjekt?
\item
  Was ist ein zugeordnete Ressource?
\item
  Können verschieden zueinander im Verhältnis stehen -- in beiden Fällen
  kann es zu Diskriminierung/Bias kommen.
\end{itemize}

\subsubsection{Verhältnis von Metadatenobjekten und zugeordneten
Ressourcen}\label{verhuxe4ltnis-von-metadatenobjekten-und-zugeordneten-ressourcen}

\paragraph{Ein Metadatenobjekt mit einer zugeordneten
Ressource}\label{ein-metadatenobjekt-mit-einer-zugeordneten-ressource}

\paragraph{Ein Metadatenobjekt mit drei zugeordneten Ressourcen
(Triptychon)}\label{ein-metadatenobjekt-mit-drei-zugeordneten-ressourcen-triptychon}

\paragraph{Ein Metadatenobjekt mit drei verschiedenen zugeordneten
Ressourcen}\label{ein-metadatenobjekt-mit-drei-verschiedenen-zugeordneten-ressourcen}

\paragraph{Zwei Metadatenobjekte mit derselben zugeordneten
Ressource}\label{zwei-metadatenobjekte-mit-derselben-zugeordneten-ressource}

\subsubsection{Standards}\label{standards}

\begin{itemize}
\tightlist
\item
  Festhalten: Es existieren verschiedene Standards\ldots{} Fokus hier
  (vorerst) auf Dublin Core
\end{itemize}

\paragraph{Schemas / Strukturstandard}\label{schemas-strukturstandard}

\begin{itemize}
\tightlist
\item
  Dublin Core Metadata Element Set
\end{itemize}

\begin{tcolorbox}[enhanced jigsaw, colbacktitle=quarto-callout-tip-color!10!white, rightrule=.15mm, coltitle=black, left=2mm, opacitybacktitle=0.6, toptitle=1mm, title=\textcolor{quarto-callout-tip-color}{\faLightbulb}\hspace{0.5em}{Tipp der Stadt.Geschichte.Basel}, colback=white, colframe=quarto-callout-tip-color-frame, opacityback=0, titlerule=0mm, leftrule=.75mm, breakable, bottomtitle=1mm, bottomrule=.15mm, arc=.35mm, toprule=.15mm]

Verweis auf Anhang mit Tabellen der Metadaten der Stadt.Geschichte.Basel

\end{tcolorbox}

\paragraph{Formatstandard}\label{formatstandard}

\begin{itemize}
\tightlist
\item
  XML/RDF
\end{itemize}

\paragraph{Inhaltsstandard}\label{inhaltsstandard}

\begin{itemize}
\tightlist
\item
  Welche Felder wie befüllt werden (z. B. Reihenfolge Vor- und
  Nachnamen)
\item
  RSWK
\end{itemize}

\paragraph{Wertstandard}\label{wertstandard}

\begin{itemize}
\tightlist
\item
  Inhalt der Felder (z. B. GND)
\end{itemize}

\subsection{Opendata-Prinzipien}\label{opendata-prinzipien}

\subsubsection{Open Data und
Langzeitarchivierung}\label{open-data-und-langzeitarchivierung}

\begin{itemize}
\tightlist
\item
  Grundsätzliches
\item
  Zenodo und andere Aggregatoren und Publikationsorte
\end{itemize}

\subsubsection{FAIR-Prinzipien}\label{fair-prinzipien}

\begin{itemize}
\tightlist
\item
  FAIR ≠ open (in gewissen Kontexten kann Einschränkung sinnvoll sein
  und ist in FAIR eingeschlossen!)
\item
  \ldots{}
\end{itemize}

\subsubsection{CARE-Prinzipien}\label{care-prinzipien}

\begin{itemize}
\tightlist
\item
  Die CARE-Prinzipien wurde innerhlab www.rd-alliance.org gegründet.

  \begin{itemize}
  \tightlist
  \item
    C:
  \item
    A:
  \item
    R:
  \item
    E:
  \end{itemize}
\item
  Stand Umsetzung aktuell

  \begin{itemize}
  \tightlist
  \item
    Noch relativ neu und kommen deshalb erst nach und nach in Umsetzung.
  \item
    Ihre hohe Anziehungskraft (gerade als Komplement zu den
    FAIR-Prinzipien) verweist jedoch auf ein grosses Bedürfnis.
  \end{itemize}
\item
  Fokus:

  \begin{itemize}
  \tightlist
  \item
    CARE-Prinzipien adressieren dezidiert indigene Daten, bietet jedoch
    keinen allgemeinen Rahmen für ethischen Fragen in Bezug auf
    Sammlungsdaten.
  \item
    Entsprechend stellt sich die Frage, inwieweit die CARE-Prinzipien
    ``fliessend'' auf andere Kontexte ausgeweitet werden können, oder ob
    die Formulierung separater Prinzipien je nach Kontext sinnvoller
    wäre.
  \item
    Die CARE-Prinzipien liefern wichtige Impulse für FDM um ethische
    Fragen im Umgang mit Forschungsdaten und -materialien zu
    reflektieren und systematisch zu berücksichtigen.
  \end{itemize}

  \paragraph{Communities}\label{communities}

  \begin{itemize}
  \tightlist
  \item
    Bei der konkreten Umsetzung stellen sich Fragen wie: ``Wer sind
    genau die \emph{Communities}?'' oder ``Wie genau bemessen wir
    \emph{collective benefit}?'' Je nach Kontext sind diese Fragen
    schwer zu beantworten.

    \begin{itemize}
    \tightlist
    \item
      Kritische Einordnung des oftmals pauschalisierend eingesetzten
      Community-Begriffs (vgl. z.B. Kaiser et al.~2023)
    \item
      Verliert eine Gruppe irgendwann die Rechte über ihre Vorfahren?
      Wer bestimmt, woran festmachen?
    \item
      \ldots{}
    \end{itemize}
  \item
    Beispiele nennen für andere Kontexte, e.g.~Datenschutzthematik,
    NS-Register, heterogene Community, \ldots{}
  \item
    TK-Labels (Fokussieren auf Selbstbestimmung bei der Label-Vergabe im
    Kontext von Institutions-Gatekeeping)
  \end{itemize}

  \paragraph{Diskriminierungsfrei
  vs.~Diskriminierungssensibel}\label{diskriminierungsfrei-vs.-diskriminierungssensibel}

  \begin{itemize}
  \tightlist
  \item
    ``Diskriminierungsfrei'' im alten Titel aufgreifen

    \begin{itemize}
    \tightlist
    \item
      Mit dieser Differenzierung soll kein Relativismus betrieben
      werden!

      \begin{itemize}
      \tightlist
      \item
        Streben hin zu Diskriminierungsfreiheit: ja, unbedingt;
        Diskriminierungsfreiheit behaupten: nein
      \end{itemize}
    \item
      Begründung

      \begin{itemize}
      \tightlist
      \item
        Diskriminierungsverständnis ist geografisch und
        communityabhängig heterogen
      \item
        Diskriminierungsverständnis ändert sich -- auch in Zukunft
      \item
        Auch ``Normdaten'' verändern sich (auch die GND)
      \item
        Dekolonialisierung nie abgeschlossen
      \item
        Archiv-Lücken betonen
      \item
        Archiv-Geschichte bleibt
      \item
        Weit mehr als bloss Vokabular
      \end{itemize}
    \end{itemize}
  \end{itemize}
\end{itemize}

\section{Praxis: Diskriminierungssensibler Umgang mit
Metadaten}\label{praxis-diskriminierungssensibler-umgang-mit-metadaten}

\subsection{Erstellung von Metadaten}\label{erstellung-von-metadaten}

\begin{itemize}
\tightlist
\item
  Abgrenzen (1) Primärerschliessung, (2) Arbeiten mit Subsets
  bestehender Metadaten -\textgreater{} in der Praxis oft eine Mischung
  aus beidem.
\item
  Folgende Arbeitsschritte sind dabei stets wichtig: Inhaltsangabe
  Folgekapitel
\end{itemize}

\subsubsection{Forschungsrahmen
abstecken}\label{forschungsrahmen-abstecken}

\begin{itemize}
\tightlist
\item
  Was beschreibe ich?
\item
  Umfang der Metadatenerfassung (Ziel, Produkt,
  Ressourcenmanagement~(Notwendiges vs.~nice to have, etc., für wen
  (Zielgruppen)?)
\item
  Prioritätensetzung und Zeitmanagement\\
\item
  Erfassen und Identifizieren der Ressource(n)
\item
  Welche Informationen werden benötigt, um die Ressourcen in den Kontext
  zu setzen
\end{itemize}

\subsubsection{Umgang mit bestehenden
Metadaten}\label{umgang-mit-bestehenden-metadaten}

\begin{itemize}
\tightlist
\item
  \textbf{Situationen:} Datenrecherche mit
  Archivkatalog/Aggregator/\ldots{}
\item
  \textbf{Woher stammen Metadaten:} Archive, Literatur, \ldots{}
\item
  \textbf{Herausforderungen:} Heterogenität, Umgang mit bestehenden
  Diskriminierungen
\item
  \textbf{Strategien:} Zusammenführung vs.~Anreicherung vs.~Ausdünnung
  vs.~komplett neu vs.~\ldots{}
\end{itemize}

\begin{tcolorbox}[enhanced jigsaw, colbacktitle=quarto-callout-tip-color!10!white, rightrule=.15mm, coltitle=black, left=2mm, opacitybacktitle=0.6, toptitle=1mm, title=\textcolor{quarto-callout-tip-color}{\faLightbulb}\hspace{0.5em}{Erfahrungen von Stadt.Geschichte.Basel}, colback=white, colframe=quarto-callout-tip-color-frame, opacityback=0, titlerule=0mm, leftrule=.75mm, breakable, bottomtitle=1mm, bottomrule=.15mm, arc=.35mm, toprule=.15mm]

Beispiele

\end{tcolorbox}

\subsubsection{Kontextualisierung der
Ressource}\label{kontextualisierung-der-ressource}

\begin{itemize}
\tightlist
\item
  Entstehungskontext der Ressource

  \begin{itemize}
  \tightlist
  \item
    Sozialer und politischer Kontext\\
  \item
    Quellenbeschreibung\\
  \item
    Kontexte der Primär-Autor:innen
  \end{itemize}
\item
  Verwendungskontext der Ressource

  \begin{itemize}
  \tightlist
  \item
    In welchem Kontext/Umfang/Zustand/Verschlagwortung wird sie heute
    zur Verfügung gestellt?
  \item
    Kontexte der betreuenden Archivar:innen etc.
  \item
    Interpretation und Rezeption in der Forschung usw. (beeinflusst etwa
    Auffindbarkeit)
  \end{itemize}
\item
  Praxis: Reflexion und Dokumentation von obigem
\end{itemize}

\begin{tcolorbox}[enhanced jigsaw, colbacktitle=quarto-callout-tip-color!10!white, rightrule=.15mm, coltitle=black, left=2mm, opacitybacktitle=0.6, toptitle=1mm, title=\textcolor{quarto-callout-tip-color}{\faLightbulb}\hspace{0.5em}{Beispiele}, colback=white, colframe=quarto-callout-tip-color-frame, opacityback=0, titlerule=0mm, leftrule=.75mm, breakable, bottomtitle=1mm, bottomrule=.15mm, arc=.35mm, toprule=.15mm]

Diverse Beispiele

\end{tcolorbox}

\subsubsection{Metadatenfelder
festlegen}\label{metadatenfelder-festlegen}

\begin{itemize}
\tightlist
\item
  Was sind die Optionen?
\item
  Identifizieren relevanter Felder für die Bedürfnisse des Projekts
\item
  Übernehmen bestehender Felder
\item
  Ergänzen bestehender Felder mit anderen Standards und Entscheidungen
  dokumentieren
\item
  Extrinsisch vs.~intrinsisch (vgl. auch Wertstandards)
\end{itemize}

\begin{tcolorbox}[enhanced jigsaw, colbacktitle=quarto-callout-tip-color!10!white, rightrule=.15mm, coltitle=black, left=2mm, opacitybacktitle=0.6, toptitle=1mm, title=\textcolor{quarto-callout-tip-color}{\faLightbulb}\hspace{0.5em}{Dublin Core}, colback=white, colframe=quarto-callout-tip-color-frame, opacityback=0, titlerule=0mm, leftrule=.75mm, breakable, bottomtitle=1mm, bottomrule=.15mm, arc=.35mm, toprule=.15mm]

\ldots{}

\end{tcolorbox}

\subsubsection{Wertstandards wählen und
verschlagworten}\label{wertstandards-wuxe4hlen-und-verschlagworten}

\begin{itemize}
\tightlist
\item
  Was sind die Optionen?
\item
  Wertstandard wählen und mit Ressourcen abgleichen
\item
  Anpassungen abwägen und vornehmen und Entscheidungen dokumentieren
\item
  Mappings: Verknüpfen verschiedener Wertstandards / Äquivalenzklassen
\item
  Extrinsisch vs.~intrinsisch (vgl. auch Metadatenfelder)
\end{itemize}

\begin{tcolorbox}[enhanced jigsaw, colbacktitle=quarto-callout-tip-color!10!white, rightrule=.15mm, coltitle=black, left=2mm, opacitybacktitle=0.6, toptitle=1mm, title=\textcolor{quarto-callout-tip-color}{\faLightbulb}\hspace{0.5em}{Limitationen}, colback=white, colframe=quarto-callout-tip-color-frame, opacityback=0, titlerule=0mm, leftrule=.75mm, breakable, bottomtitle=1mm, bottomrule=.15mm, arc=.35mm, toprule=.15mm]

\begin{itemize}
\tightlist
\item
  Das Verwenden und Ausgeben von Objektbezeichnungen in einer
  Lokalsprache setzt voraus, dass eine 1:1-Übersetzung existiert, was
  nicht immer der Fall ist.
\end{itemize}

\end{tcolorbox}

\begin{tcolorbox}[enhanced jigsaw, colbacktitle=quarto-callout-note-color!10!white, rightrule=.15mm, coltitle=black, left=2mm, opacitybacktitle=0.6, toptitle=1mm, title=\textcolor{quarto-callout-note-color}{\faInfo}\hspace{0.5em}{Ausschlüsse durch Mapping}, colback=white, colframe=quarto-callout-note-color-frame, opacityback=0, titlerule=0mm, leftrule=.75mm, breakable, bottomtitle=1mm, bottomrule=.15mm, arc=.35mm, toprule=.15mm]

\begin{itemize}
\tightlist
\item
  kumulative Äquivalenz
\end{itemize}

\end{tcolorbox}

\begin{tcolorbox}[enhanced jigsaw, colbacktitle=quarto-callout-note-color!10!white, rightrule=.15mm, coltitle=black, left=2mm, opacitybacktitle=0.6, toptitle=1mm, title=\textcolor{quarto-callout-note-color}{\faInfo}\hspace{0.5em}{Stolpersteine bei Begriffsauswahl}, colback=white, colframe=quarto-callout-note-color-frame, opacityback=0, titlerule=0mm, leftrule=.75mm, breakable, bottomtitle=1mm, bottomrule=.15mm, arc=.35mm, toprule=.15mm]

\begin{itemize}
\tightlist
\item
  Sprachspezifische Unterschiede (Rasse/Race)
\item
  Kontext (Zwerg)
\item
  Wiederaneignungen und Selbstbeschreibungen (Crip)
\item
  Stehende Begriffe (e.g.~\emph{Lycaon krebsi})
\item
  \ldots{}
\end{itemize}

\end{tcolorbox}

\begin{tcolorbox}[enhanced jigsaw, colbacktitle=quarto-callout-tip-color!10!white, rightrule=.15mm, coltitle=black, left=2mm, opacitybacktitle=0.6, toptitle=1mm, title=\textcolor{quarto-callout-tip-color}{\faLightbulb}\hspace{0.5em}{Stolpersteine bei der Verschlagwortung}, colback=white, colframe=quarto-callout-tip-color-frame, opacityback=0, titlerule=0mm, leftrule=.75mm, breakable, bottomtitle=1mm, bottomrule=.15mm, arc=.35mm, toprule=.15mm]

\begin{itemize}
\tightlist
\item
  Umgang mit Darstellungen von Nacktheit.
  Malegaze/Objektivierung/Sexualisierung/Prüdheit/Selbstbestimmung/\ldots{}
  (``Bad zu Leuk'', \ldots)
\item
  Werte-Urteile von aussen ohne Insider-Wissen
\item
  Gut gemeint aber\ldots{}

  \begin{itemize}
  \tightlist
  \item
    Neue Diskriminierungen schaffen durch\ldots{}

    \begin{itemize}
    \tightlist
    \item
      Monofokus (Keltin)\\
    \item
      Alarmglocken (Nacktheit)\\
    \end{itemize}
  \end{itemize}
\item
  Übersetzungen (vgl. oben)\\
\end{itemize}

\end{tcolorbox}

\begin{tcolorbox}[enhanced jigsaw, colbacktitle=quarto-callout-tip-color!10!white, rightrule=.15mm, coltitle=black, left=2mm, opacitybacktitle=0.6, toptitle=1mm, title=\textcolor{quarto-callout-tip-color}{\faLightbulb}\hspace{0.5em}{`Neue' Konzepte und historische Quellen}, colback=white, colframe=quarto-callout-tip-color-frame, opacityback=0, titlerule=0mm, leftrule=.75mm, breakable, bottomtitle=1mm, bottomrule=.15mm, arc=.35mm, toprule=.15mm]

\begin{itemize}
\tightlist
\item
  Beim Verschlagworten und Beschreiben historischer Quellen sollten
  Begriffe stets so akkurat wie möglich gewählt werden.
\item
  Zuweilen wird dieses Argument jedoch verwendet, um bspw. die
  Auszeichnung potenziell nichtbinärer Menschen als ideologisch oder
  unwissenschaftlich zu diskreditieren. Nichtbinarität sei ein
  zeitgenössisches Konzept, heisst es dann, welches in historischen
  Daten nichts verloren habe.
\item
  Dekonstruktion dieser Behauptung.
\end{itemize}

\end{tcolorbox}

\begin{tcolorbox}[enhanced jigsaw, colbacktitle=quarto-callout-tip-color!10!white, rightrule=.15mm, coltitle=black, left=2mm, opacitybacktitle=0.6, toptitle=1mm, title=\textcolor{quarto-callout-tip-color}{\faLightbulb}\hspace{0.5em}{Systematische Lücken bei Schlagwortindexen}, colback=white, colframe=quarto-callout-tip-color-frame, opacityback=0, titlerule=0mm, leftrule=.75mm, breakable, bottomtitle=1mm, bottomrule=.15mm, arc=.35mm, toprule=.15mm]

(ggf. in Kapitel \emph{2-Schlüsselbegriffe} verschieben) - Beispiele:
Gender-Open-Pendent für Kolonialitäts-Kontext, Gender-Open-Pendent
e.g.~für Italienisch - Gründe für Fehlen: \ldots{} - Bedarf und Chancen:
\ldots{} - Hürden: \ldots{}

\end{tcolorbox}

\begin{tcolorbox}[enhanced jigsaw, colbacktitle=quarto-callout-tip-color!10!white, rightrule=.15mm, coltitle=black, left=2mm, opacitybacktitle=0.6, toptitle=1mm, title=\textcolor{quarto-callout-tip-color}{\faLightbulb}\hspace{0.5em}{Sollen wir eine eigene Ontologie erstellen?}, colback=white, colframe=quarto-callout-tip-color-frame, opacityback=0, titlerule=0mm, leftrule=.75mm, breakable, bottomtitle=1mm, bottomrule=.15mm, arc=.35mm, toprule=.15mm]

\begin{itemize}
\tightlist
\item
  Erfahrungen der Stadt.Geschichte.Basel zum Einsatz von OpenGender

  \begin{itemize}
  \tightlist
  \item
    nur OpenGender Schlagwortindex?
  \item
    interne Strategie
  \end{itemize}
\item
  Chancen: \ldots{}
\item
  Gefahren: \ldots{}
\item
  Notwendigkeit: es fehlen viele Indexe (vgl. Box ``systematische Lücken
  bei Schlagwortindexen'')
\item
  Strategien und Alternativen

  \begin{itemize}
  \tightlist
  \item
    Kombination/Mapping verschiedener Indexe (vgl. nächste Box)
  \item
    Subset erstellen
  \end{itemize}
\end{itemize}

\end{tcolorbox}

\begin{tcolorbox}[enhanced jigsaw, colbacktitle=quarto-callout-tip-color!10!white, rightrule=.15mm, coltitle=black, left=2mm, opacitybacktitle=0.6, toptitle=1mm, title=\textcolor{quarto-callout-tip-color}{\faLightbulb}\hspace{0.5em}{Best Practices für Mappings}, colback=white, colframe=quarto-callout-tip-color-frame, opacityback=0, titlerule=0mm, leftrule=.75mm, breakable, bottomtitle=1mm, bottomrule=.15mm, arc=.35mm, toprule=.15mm]

\begin{itemize}
\tightlist
\item
  Tipps, Ressourcen und Aggregatoren
\item
  GND-OpenGender-Mapping
\end{itemize}

\end{tcolorbox}

\subsubsection{Transparente Metadaten: Versionierung und
Dokumentation}\label{transparente-metadaten-versionierung-und-dokumentation}

\begin{itemize}
\tightlist
\item
  Relevanz

  \begin{itemize}
  \tightlist
  \item
    Wer hat wann und wieso welche Metadaten erstellt/geändert?
  \item
    \emph{``Wie dokumentieren wir, wie wir selber erschliessen? (In 50
    Jahren wird auch das wieder interessieren und in Frage gestellt
    werden.)''} Tobias Wildi (Rückmeldung per Linkedin)
  \item
    \emph{``Veraltete Metadaten sollen nicht einfach gelöscht, sondern
    durch neue aktuelle Beschreibungen überlagert werden. Es muss
    möglich sein, die Entwicklung verschiedener Generationen von
    Metadaten über die Zeit nachzuverfolgen, denn an ihnen manifestiert
    sich, wie wir Dinge benennen und über sie denken. Und die Welt steht
    wie gesagt nicht still, die heute angefertigten Metadaten werden
    nicht die letzten sein. Zukünftige Generationen werden ihrerseits
    aus unseren Metadaten auf unsere gesellschaftlichen Verhältnisse und
    Wertvorstellen rückschliessen. Diese Nachvollziehbarkeit ist aber
    nur möglich, wenn alte Metadaten nicht gelöscht, sondern versioniert
    werden und ihr Entstehungskontext dokumentiert wird.''} Tobias Wildi
    (Rückmeldung per Linkedin)
  \item
    Historische Nachvollziehbarkeit: \emph{``Unter welchen Umständen und
    durch welche Personen sind Metadaten mit Begrifflichkeiten und
    Weltvorstellungen entstanden, von denen wir uns heute abgrenzen?''}
    Tobias Wildi (Rückmeldung per Linkedin)
  \end{itemize}
\item
  Umsetzung

  \begin{itemize}
  \tightlist
  \item
    Technische Integration in bestehende Sammlungsmanagement-Tools
  \item
    Automatisierbarkeit
  \end{itemize}
\item
  Abwägungen

  \begin{itemize}
  \tightlist
  \item
    Detailgrad
  \item
    Sichtbarkeit nach aussen (siehe unten)
  \end{itemize}
\end{itemize}

\begin{tcolorbox}[enhanced jigsaw, colbacktitle=quarto-callout-tip-color!10!white, rightrule=.15mm, coltitle=black, left=2mm, opacitybacktitle=0.6, toptitle=1mm, title=\textcolor{quarto-callout-tip-color}{\faLightbulb}\hspace{0.5em}{Beispiel für technische/konzeptuelle Umsetzung Versionierung}, colback=white, colframe=quarto-callout-tip-color-frame, opacityback=0, titlerule=0mm, leftrule=.75mm, breakable, bottomtitle=1mm, bottomrule=.15mm, arc=.35mm, toprule=.15mm]

\begin{itemize}
\tightlist
\item
  \ldots{}
\item
  Mindestanforderungen

  \begin{itemize}
  \tightlist
  \item
    Zugriff via (Volltext)suche (vgl. unten)
  \item
    ``This content was machine translated''
  \end{itemize}
\end{itemize}

\end{tcolorbox}

\begin{tcolorbox}[enhanced jigsaw, colbacktitle=quarto-callout-tip-color!10!white, rightrule=.15mm, coltitle=black, left=2mm, opacitybacktitle=0.6, toptitle=1mm, title=\textcolor{quarto-callout-tip-color}{\faLightbulb}\hspace{0.5em}{Beispiel für technische/konzeptuelle Umsetzung Dokumentation}, colback=white, colframe=quarto-callout-tip-color-frame, opacityback=0, titlerule=0mm, leftrule=.75mm, breakable, bottomtitle=1mm, bottomrule=.15mm, arc=.35mm, toprule=.15mm]

\begin{itemize}
\tightlist
\item
  \ldots{}
\end{itemize}

\end{tcolorbox}

\begin{tcolorbox}[enhanced jigsaw, colbacktitle=quarto-callout-tip-color!10!white, rightrule=.15mm, coltitle=black, left=2mm, opacitybacktitle=0.6, toptitle=1mm, title=\textcolor{quarto-callout-tip-color}{\faLightbulb}\hspace{0.5em}{Wohin mit der Dokumentation}, colback=white, colframe=quarto-callout-tip-color-frame, opacityback=0, titlerule=0mm, leftrule=.75mm, breakable, bottomtitle=1mm, bottomrule=.15mm, arc=.35mm, toprule=.15mm]

\begin{itemize}
\tightlist
\item
  Gehört eine Dokumentation in jede einzelne Ressource oder reicht es,
  eine zentrale Dokumentation für einen ganzen Datensatz anzulegen?
\item
  Wie kann nebst den Daten auch die Dokumentation interoperabel
  gestaltet werden?
\end{itemize}

\end{tcolorbox}

\subsubsection{Umgang mit Nicht-Wissen}\label{umgang-mit-nicht-wissen}

\paragraph{Lücken und Unsicherheiten in
Quellen}\label{luxfccken-und-unsicherheiten-in-quellen}

\begin{itemize}
\tightlist
\item
  Sammlungslücken und fehlende Metadaten/Informationen sind auch eine
  wichtige Information!

  \begin{itemize}
  \tightlist
  \item
    Beispiele aus e.g.~Militär-Archiven sowie aus Provenienzforschung
    bringen.
  \item
    Problematik der Unsichtbarmachung im Archiv (hier bloss Verweis auf
    Schlüsselbegriffe)
  \end{itemize}
\item
  Leerstellen: Zum Umgang beim Leerlassen von Metadatenfeldern:

  \begin{itemize}
  \tightlist
  \item
    ggf. sollte Absicht hinter dem Leerlassen dokumentiert werden, denn
    Info existiert nicht ≠ Info nicht gefunden ≠ Datensatz nicht/nur
    teilweise bearbeitet.
  \item
    analoges gilt für Tagging von Beständen als un/problematisch.
  \end{itemize}
\item
  Qualifikation der Metadaten: Zum Umgang mit Unsicherheiten:

  \begin{itemize}
  \tightlist
  \item
    z.B. unsichere oder unleserliche Quelle ≠ widersprüchliche,
    fragwürdie oder umstrittene Quellenlage
  \end{itemize}
\end{itemize}

\paragraph{Expertisen einholen und
einbinden}\label{expertisen-einholen-und-einbinden}

\begin{itemize}
\tightlist
\item
  Expertisen identifizieren.

  \begin{itemize}
  \tightlist
  \item
    Wer hat Autorität, (mit)zubestimmen? (vgl.
    \emph{2-Schlüsselbegriffe/Communities})
  \end{itemize}
\item
  Eigene Positioniertheit reflektieren
\item
  Wechselseitige Machtdynamiken, Interessen, Bedürfnisse und
  Verletzlichkeiten reflektieren (auf beiden Seiten)
\item
  Potenziale, Formen (auch Grenzen) von Dialog und Mitbestimmung
  gemeinsam ausloten
\item
  Settings herstellen um Expertise optimal einzuholen
\item
  Umgang mit sich widersprechenden Sichtweisen bzw. Bedürfnissen (ggf.
  auch innerhalb einer ``Community'')

  \begin{itemize}
  \tightlist
  \item
    Zur Problematik des oftmals pauschalisierend eingesetzten
    Community-Begriffs, siehe \emph{2-Schlüsselbegriffe/Communities}
  \item
    Interessensabwägung
  \item
    Dokumentieren in Freitextfeld
  \item
    \ldots{}
  \end{itemize}
\end{itemize}

\subsubsection{Umgang mit LLMs}\label{umgang-mit-llms}

(Nicht im Fokus von Handbuch 2.0 → Verweisen auf Handbuch 3.0)

\begin{itemize}
\tightlist
\item
  Dringlichkeit, dieses Kapitel ins Handbuch aufzunehmen

  \begin{itemize}
  \tightlist
  \item
    Vorstellung ist verbreitet, AI könne Verschlagwortung übernehmen
  \item
    Sensibilisierung für Biases in Trainingsdaten
  \item
    Metadaten für LLM optimiert publizieren als Teil von FAIR-Data?
    (machinereadable/LLM-readable data)
  \item
    Umgang mit bereits bestehenden AI-generierten Metadaten

    \begin{itemize}
    \tightlist
    \item
      wie erkennen?
    \item
      worauf achten?
    \item
      wie markieren/dokumentieren?
    \end{itemize}
  \end{itemize}
\item
  Publizieren von Metadaten
\item
  AI-Tools

  \begin{itemize}
  \tightlist
  \item
    Diskussion bestehender Tools

    \begin{itemize}
    \tightlist
    \item
      DE-BIAS-Tool
    \item
      \ldots{}
    \end{itemize}
  \item
    Chancen von und Anforderungen an AI-Tools

    \begin{itemize}
    \tightlist
    \item
      man-in-the-loop
    \end{itemize}
  \item
    Grenzen von AI-Tools

    \begin{itemize}
    \tightlist
    \item
      Umgang mit Halluzinationen.
    \item
      Autor:innenschaft / Haftungsprobleme mit vollautomatisch
      eingesetzter KI
    \end{itemize}
  \end{itemize}
\end{itemize}

\begin{tcolorbox}[enhanced jigsaw, colbacktitle=quarto-callout-tip-color!10!white, rightrule=.15mm, coltitle=black, left=2mm, opacitybacktitle=0.6, toptitle=1mm, title=\textcolor{quarto-callout-tip-color}{\faLightbulb}\hspace{0.5em}{Positive Erfahrungen aus der Praxis}, colback=white, colframe=quarto-callout-tip-color-frame, opacityback=0, titlerule=0mm, leftrule=.75mm, breakable, bottomtitle=1mm, bottomrule=.15mm, arc=.35mm, toprule=.15mm]

\begin{itemize}
\tightlist
\item
  Beispiele sammeln
\end{itemize}

\end{tcolorbox}

\begin{tcolorbox}[enhanced jigsaw, colbacktitle=quarto-callout-tip-color!10!white, rightrule=.15mm, coltitle=black, left=2mm, opacitybacktitle=0.6, toptitle=1mm, title=\textcolor{quarto-callout-tip-color}{\faLightbulb}\hspace{0.5em}{Negative Erfahrungen aus der Praxis}, colback=white, colframe=quarto-callout-tip-color-frame, opacityback=0, titlerule=0mm, leftrule=.75mm, breakable, bottomtitle=1mm, bottomrule=.15mm, arc=.35mm, toprule=.15mm]

\begin{itemize}
\tightlist
\item
  Beispiele sammeln
\end{itemize}

\end{tcolorbox}

\subsection{Zugang und Publikation}\label{zugang-und-publikation}

\begin{itemize}
\tightlist
\item
  Metadaten machen Anschein, alles sei suchbar und da. Doch nein!
\item
  Bei Publikation von Metadaten muss stets auch Rezeptionsebene
  mitgedacht werden -- ganz besonders z.B. auf Sammlungsportalen.
\item
  Im Kontext sensibler Daten geht es dabei auch um Fragen von Zugang
\item
  Und im Kontext von Open Data und Aggregatoren auch um die Frage,
  inwiefern eine diskriminierungssensible Weitergabe und Präsentation
  von Daten gefördert werden kann.
\end{itemize}

\subsubsection{Rahmen abstecken}\label{rahmen-abstecken}

\begin{itemize}
\tightlist
\item
  Metadaten als Findhilfe? Oder mehr?
\end{itemize}

\subsubsection{Umgang mit sensiblen und/oder diskriminierenden
Inhalten}\label{umgang-mit-sensiblen-undoder-diskriminierenden-inhalten}

\begin{itemize}
\tightlist
\item
  \textbf{Was:} Sensible vs.~diskriminierende Daten
\item
  \textbf{Wirkung:}

  \begin{itemize}
  \tightlist
  \item
    Trigger vermeiden vs.
  \item
    Schutz vulnerabler Einzelpersonen/Menschengruppen/Kulturgut vs.~
  \item
    Reproduktion diskriminierender Narrative
  \end{itemize}
\item
  \textbf{Ebenen:} Rechtslage vs.~Ethik (es geht um Sicherheit, aber
  auch um Würde)
\item
  \textbf{Abwägung:} Potenzieller Intersssenskonflikt zwischen (1)
  Zugang erleichtern und (2) Schaden vermeiden (nicht immer einfach)
\end{itemize}

\paragraph{Identifizieren sensibler
Daten}\label{identifizieren-sensibler-daten}

\begin{itemize}
\tightlist
\item
  Beispiele

  \begin{itemize}
  \tightlist
  \item
    \textbf{Geodaten:} So kann es etwa Sinn machen, Fundorte
    archäologischer Objekte nicht öffentlich zu machen, um Risiko von
    Plünderungen zu reduzieren\\
  \item
    \textbf{Personendaten im Archiv:} sensible Informationen
    lebender/toter Personen Objekte (z.B.
    Aufenthaltsorte/Interviewtranskripte mit Nachfahren, Zeitzeug:innen,
    \ldots)
  \item
    \textbf{Personendaten der Archivar:innen:} Transparenz liefert
    wohlverdiente Sichtbarkeit für Sachbearbeiter:innen, exponiert sie
    aber gleichzeitig auch. Ggf. Kürzel statt Klarnamen verwenden
  \end{itemize}
\item
  Rechtliche/Ethische Grundsätze

  \begin{itemize}
  \tightlist
  \item
    \ldots{}
  \end{itemize}
\end{itemize}

\paragraph{Identifizieren diskriminierender
Daten}\label{identifizieren-diskriminierender-daten}

\begin{itemize}
\tightlist
\item
  Beispiele

  \begin{itemize}
  \tightlist
  \item
    \textbf{Diskriminierende Beschreibungen:} \ldots{}
  \item
    \textbf{Zuschreibungen und Stereotypisierungen:} \ldots{}
  \end{itemize}
\item
  Rechtliche/Ethische Grundsätze

  \begin{itemize}
  \tightlist
  \item
    \ldots{}
  \end{itemize}
\end{itemize}

\paragraph{Strategien zum
Zugangs-Management}\label{strategien-zum-zugangs-management}

\begin{itemize}
\tightlist
\item
  Grundsätzlich: Was zeigen wir wem wie weshalb und unter welchen
  Bedingungen?

  \begin{itemize}
  \tightlist
  \item
    Betrifft ein Datenportal genauso wie dieses Handbuch
  \item
    \textbf{Wer:} Unterschiedliche Zugänge für verschiedene Zielpublika
    (Menschen mit spezifischer Forschungsfrage, Online-Datenbank, Dialog
    mit ``Herkunftsgesellschaften'', Web-Crawlers, \ldots)
  \item
    \textbf{Worauf:} Vollzugriff vs.~Kontextualisierung vs.~Teilzugriff
    vs.~Versionsgeschichte
  \item
    \textbf{Kontextualisierung:} Unter welcher Frage/Suchbegriffen soll
    ein bestimmter Inhalt (z.B. ein Bild) gefunden werden? (z.B.
    ``Antisemitismus'' oder ``jüdische Menschen''?)
  \end{itemize}
\item
  \textbf{Umgang mit Bildern}

  \begin{itemize}
  \tightlist
  \item
    Relevant für Portale, Ausstellungen, Slides auf Vorträgen
  \item
    Entsprechende Überlegungen sollten auch an einer Fachtagungen
    gemacht werden (z.B. beim zeigen rassistischer Praktiken auf
    Bildern)
  \end{itemize}
\item
  \textbf{Umgang mit Text}

  \begin{itemize}
  \tightlist
  \item
    Strategien: Verfremdung, Verzerrung, Durchstreichung,
    Sternchenbarriere (N***), Änderung, Anführungszeichen
  \item
    Diskussion: Rücksicht mit historischem Kontext
  \item
    Diskussion zu Titeln
  \end{itemize}
\item
  \textbf{Umgang mit Biased Perspektiven}

  \begin{itemize}
  \tightlist
  \item
    Reproduktion von Blickregimen (z.B. in der
    \href{https://www.deutschefotothek.de/cms/weltsichten.xml}{Deutschen
    Fotothek})
  \item
    Auswahl von Touren / Eingangsnarrativen / Kontextualisierungen (und
    Kontextualisierungsprinzipien!!) / bewussten Umwegen /
    Bildminiaturen
  \end{itemize}
\item
  \textbf{Umgang mit Lücken}

  \begin{itemize}
  \tightlist
  \item
    Explizieren, dass bspw. Namen von Personen unbekannt (bzw. nicht
    archiviert/notiert/als erwähnenswert befunden worden)
  \end{itemize}
\item
  \textbf{Zugangsbeschränkung}

  \begin{itemize}
  \tightlist
  \item
    API-Key als (symbolische) Hürde
  \item
    Content-Notes/Disclaimer

    \begin{itemize}
    \tightlist
    \item
      MARC-Feld 500
    \item
      Kommentar zu Länge, Detailgrad, Inhalt
    \item
      User wählen lassen, ob sie spezifische Inhalte sehen wollen
      (CSS-Media-Feature)
    \end{itemize}
  \end{itemize}
\item
  \textbf{Schutz von Datenbanken vor unbefugtem Zugriff}

  \begin{itemize}
  \tightlist
  \item
    Verantwortung
  \end{itemize}
\end{itemize}

\begin{tcolorbox}[enhanced jigsaw, colbacktitle=quarto-callout-tip-color!10!white, rightrule=.15mm, coltitle=black, left=2mm, opacitybacktitle=0.6, toptitle=1mm, title=\textcolor{quarto-callout-tip-color}{\faLightbulb}\hspace{0.5em}{Erfahrungen}, colback=white, colframe=quarto-callout-tip-color-frame, opacityback=0, titlerule=0mm, leftrule=.75mm, breakable, bottomtitle=1mm, bottomrule=.15mm, arc=.35mm, toprule=.15mm]

\begin{itemize}
\tightlist
\item
  Umgang mit Zensurvorwürfen beim Markieren/Kontextualisieren

  \begin{itemize}
  \tightlist
  \item
    Beispiel Stadtbücherei Münster:
    \href{https://www.rtl.de/cms/muenster-warnhinweise-in-stadtbuecherei-sorgen-fuer-diskussionen-5095758.html}{Link}
  \end{itemize}
\item
  GND

  \begin{itemize}
  \tightlist
  \item
    Umgang mit veralteten/alternativen Begriffen
  \item
    RSWK erwähnt Datenethik nicht
  \item
    Problematik Verwendungshäufigkeit: z.B. ``I---'', welches noch
    häufig verwendet wird?
  \end{itemize}
\item
  \ldots{}
\end{itemize}

\end{tcolorbox}

\begin{tcolorbox}[enhanced jigsaw, colbacktitle=quarto-callout-tip-color!10!white, rightrule=.15mm, coltitle=black, left=2mm, opacitybacktitle=0.6, toptitle=1mm, title=\textcolor{quarto-callout-tip-color}{\faLightbulb}\hspace{0.5em}{Umgang mit Zitaten und Titeln}, colback=white, colframe=quarto-callout-tip-color-frame, opacityback=0, titlerule=0mm, leftrule=.75mm, breakable, bottomtitle=1mm, bottomrule=.15mm, arc=.35mm, toprule=.15mm]

\begin{itemize}
\tightlist
\item
  Vorstellung, dass Werk/Ressource nur \emph{einen} Titel haben kann,
  hat grundsätzlich keinen ontologischen Grund. D.h. es ist durchaus
  möglich, problematische Titel zu ändern!

  \begin{itemize}
  \tightlist
  \item
    Beispiele
  \end{itemize}
\item
  Sternchenbarrieren (N***) in Titeln

  \begin{itemize}
  \tightlist
  \item
    möglich solange kommentiert und falls wissenschaftliche
    Volltextsuche auf Volltext (und ggf. Versionen) zugreifen kann.
  \end{itemize}
\end{itemize}

\end{tcolorbox}

\subsection{Handlungsfähig bleiben}\label{handlungsfuxe4hig-bleiben}

\begin{itemize}
\tightlist
\item
  \emph{``Es ist halt schlicht unmöglich, komplett diskriminierungsfrei
  zu sein.''}
\item
  \emph{``Dann können wir ja gar nichts mehr veröffentlichen''}
\item
  Plädoyer für konstruktiven und proaktiven Umgang aus der ``Komplett
  biasfreie Daten sind doch eh nicht möglich''-Blockade raus, ohne
  Extremszenarien gegeneinander auszuspielen.
\item
  Gleichzeitig Warnung vor ``es sich mit Kompromissen zu einfach
  machen''.
\end{itemize}

\subsubsection{Typische Widerstände und
Konflikte}\label{typische-widerstuxe4nde-und-konflikte}

\begin{itemize}
\tightlist
\item
  Es kann herausfordernd und persönlich belastend sein, Widerstand zu
  leisten und kritische Positionen konstruktiv einzubringen.

  \begin{itemize}
  \tightlist
  \item
    Dieses Kapitel des Handbuchs dient dem Teilen oft gehörter
    Widerstände und Konflikte, und Strategien, wie diese aufgelöst
    werden können.
  \end{itemize}
\end{itemize}

\paragraph{Widerstände im
Team/Institution}\label{widerstuxe4nde-im-teaminstitution}

\begin{quote}
``Komplett biasfreie Daten sind doch eh nicht möglich''
\end{quote}

\begin{quote}
``2015 waren die Diskurse noch nicht da''
\end{quote}

\begin{quote}
``Ich finde es ja toll, was ihr hier zu Datenethik diskutiert. Aber in
der Praxis uns fehlt einfach die Zeit dazu''
\end{quote}

\begin{quote}
``Welche Missbrauchsfälle sind euch bekannt?''
\end{quote}

\begin{itemize}
\tightlist
\item
  Falsche Frage, wenn es um Reproduktion von Narrativen geht. Hier geht
  es nicht um böse Absichten, sondern um schleichende aber machtvolle
  Prozesse. Entsprechend sollten nicht Einzelfälle gesucht (und nicht
  gefunden) werden.
\end{itemize}

\begin{quote}
``Wir wissen nicht, wen wir fragen sollten. Und die
Ursprungsgemeinschaften erreichen wir ja eh nicht.''
\end{quote}

\begin{quote}
``Das ist technisch halt nicht möglich''
\end{quote}

\begin{quote}
``Wir haben jetzt schon zu viele Metadatenfelder.''
\end{quote}

\begin{quote}
``Gender ist ein neues Konzept und gehört nicht in historischen
Metadaten''
\end{quote}

\begin{itemize}
\tightlist
\item
  vgl. Box oben in ``Erstellen von Metadaten''
\end{itemize}

\begin{quote}
``Als Museum sind wir der Objektivität verpflichtet. Wir machen keinen
politischen Aktivismus''
\end{quote}

\begin{quote}
``Als Doktorand:in finde ich es abschreckend, eine lange Checkliste
vorgesetzt zu kriegen.''
\end{quote}

\begin{itemize}
\tightlist
\item
  CARE-Prinzipien als anregende Frage- und Reflexionsliste einführen
\end{itemize}

\begin{tcolorbox}[enhanced jigsaw, colbacktitle=quarto-callout-note-color!10!white, rightrule=.15mm, coltitle=black, left=2mm, opacitybacktitle=0.6, toptitle=1mm, title=\textcolor{quarto-callout-note-color}{\faInfo}\hspace{0.5em}{Wir sammeln weitere Statements}, colback=white, colframe=quarto-callout-note-color-frame, opacityback=0, titlerule=0mm, leftrule=.75mm, breakable, bottomtitle=1mm, bottomrule=.15mm, arc=.35mm, toprule=.15mm]

\begin{itemize}
\tightlist
\item
  bitte melden per Email oder Pullrequest
\end{itemize}

\end{tcolorbox}

\paragraph{Konflikte auf
Förderebene}\label{konflikte-auf-fuxf6rderebene}

\begin{quote}
``Wir hätten Projektfinanzierung nicht erhalten, hätten wir die Bilder
nicht vollumfassend offen publiziert.''
\end{quote}

\begin{quote}
``Wir brauchen ein Endprodukt''
\end{quote}

\begin{quote}
``Uns sind die Hände gebunden''
\end{quote}

\begin{tcolorbox}[enhanced jigsaw, colbacktitle=quarto-callout-note-color!10!white, rightrule=.15mm, coltitle=black, left=2mm, opacitybacktitle=0.6, toptitle=1mm, title=\textcolor{quarto-callout-note-color}{\faInfo}\hspace{0.5em}{Wir sammeln weitere Statements}, colback=white, colframe=quarto-callout-note-color-frame, opacityback=0, titlerule=0mm, leftrule=.75mm, breakable, bottomtitle=1mm, bottomrule=.15mm, arc=.35mm, toprule=.15mm]

\begin{itemize}
\tightlist
\item
  bitte melden per Email oder Pullrequest
\end{itemize}

\end{tcolorbox}

\subsubsection{Wie weiter mit abgeschlossenen
Projekten?}\label{wie-weiter-mit-abgeschlossenen-projekten}

\begin{itemize}
\tightlist
\item
  Bestehende Plattformen werden (aktuellen) Ansprüchen zu
  diskriminierungssensiblem Umgang oft nicht gerecht.
\item
  Leider fehlt es oft an Zeit und Finanzierung, um Änderungen
  vorzunehmen. Oder es können technische, personelle oder
  institutionspolitische Hürden bestehen.
\item
  Kritische Auseinandersetzung mit bestehenden Projekte kann sehr
  produktiv sein, da es auf potenzielle Hürden zukünftiger Projekte
  sensibilisiert.
\item
  Gleichzeitig stellt sich die Frage, wie mit abgeschlossenen Projekten
  umzugehen ist. Denn wenn datenethische Grundsätze verpflichten zu
  Aktion zur Reduktion von Schaden.
\item
  Konkret kann dies heissen: \ldots{}
\end{itemize}

\section{Literatur}\label{literatur}

\phantomsection\label{refs}
\begin{CSLReferences}{1}{0}
\bibitem[\citeproctext]{ref-australianresearchdatacommonsardc2020a}
Australian Research Data Commons (ARDC). 2020. {«{ARDC Metadata
Guide}»}, März. \url{https://doi.org/10.5281/ZENODO.6459832}.

\bibitem[\citeproctext]{ref-baca2016}
Baca, Murtha. 2016. \emph{Introduction to Metadata}. Herausgegeben von
Murtha Baca. 3. Aufl. Los Angeles: Getty Publications.
\url{http://www.getty.edu/publications/intrometadata}.

\bibitem[\citeproctext]{ref-carroll2021}
Carroll, Stephanie Russo, Edit Herczog, Maui Hudson, Keith Russell, und
Shelley Stall. 2021. {«Operationalizing the {CARE} and {FAIR Principles}
for {Indigenous} Data Futures»}. \emph{Scientific Data} 8 (1): 108.
\url{https://doi.org/10.1038/s41597-021-00892-0}.

\bibitem[\citeproctext]{ref-davis2021a}
Davis, Edie, und Bahareh Heravi. 2021. {«Linked {Data} and {Cultural
Heritage}: {A Systematic Review} of {Participation}, {Collaboration},
and {Motivation}»}. \emph{Journal on Computing and Cultural Heritage} 14
(2): 1--18. \url{https://doi.org/10.1145/3429458}.

\bibitem[\citeproctext]{ref-dogtas2022}
Doğtaş, Gürsoy, Marc-Paul Ibitz, Fatima Jonitz, Veronika Kocher, Astrid
Poyer, und Laurenz Stapf. 2022. {«Kritik an rassifizierenden und
diskriminierenden Titeln und Metadaten -- Praxisorientierte
Lösungsansätze»}. \emph{027.7 Zeitschrift für Bibliothekskultur /
Journal for Library Culture} 9 (4).
\url{https://doi.org/10.21428/1bfadeb6.abe15b5e}.

\bibitem[\citeproctext]{ref-freire2019a}
Freire, Nuno, Pável Calado, und Bruno Martins. 2019. {«Availability of
{Cultural Heritage Structured Metadata} in the {World Wide Web}»}. In
\emph{Connecting the {Knowledge Commons} --- {From Projects} to
{Sustainable Infrastructure}}, herausgegeben von Leslie Chan und Pierre
Mounier, 121--33. OpenEdition Press.
\url{https://doi.org/10.4000/books.oep.9024}.

\bibitem[\citeproctext]{ref-sgg2023}
Gabay, Simon, Tobias Hodel, Moritz Mähr, Stefan Nellen, Barbara
Roth-Lochner, Pascale Sutter, Andrea Voellmin, und Karin von Wartburg.
2023. {«Datenstandards Für Die Historische {Forschung} -- {Ein
White-Paper} Der {Schweizerischen Gesellschaft} Für {Geschichte}»}.
Herausgegeben von Schweizerische Gesellschaft für Geschichte.
\emph{Whitepaper}, November.
\url{https://doi.org/10.5281/ZENODO.10122052}.

\bibitem[\citeproctext]{ref-gruber2022a}
Gruber, Andrea. 2022. {«Vom Knüpfen feministischer Begriffsnetze:
Ariadnes Faden \& geschlechtersensible Normdaten»}. \emph{Mitteilungen
der Vereinigung Österreichischer Bibliothekarinnen und Bibliothekare} 75
(1): 262--88. \url{https://doi.org/10.31263/voebm.v75i1.7213}.

\bibitem[\citeproctext]{ref-heinrich2018}
Heinrich, Andreas, und Anita Runge. 2018. {«GenderOpen: Ein Repositorium
für die Geschlechterforschung»}. \url{https://doi.org/10.25595/584}.

\bibitem[\citeproctext]{ref-jaffeb}
Jaffe, Rachel. o.~J. {«Library {Guides}: {Metadata Creation}»}. Guide.
Zugegriffen 5. Mai 2024.
\url{https://guides.library.ucsc.edu/c.php?g=618773&p=4306381}.

\bibitem[\citeproctext]{ref-lampe2021}
Lampe, Moritz. 2021. \emph{Diskriminierende Begriffe und
Wissensordnungen im Bildarchiv}. Berliner handreichungen zur
bibliotheks- und informationswissenschaft 481. Berlin: Institut für
Bibliotheks- und Informationswissenschaft der Humboldt-Universität zu
Berlin. \url{https://doi.org/10.18452/23766}.

\bibitem[\citeproctext]{ref-sparber2016a}
Sparber, Sandra. 2016. {«What's the frequency, Kenneth? -- Eine
(queer)feministische Kritik an Sexismen und Rassismen im
Schlagwortkatalog»}. \emph{Mitteilungen der Vereinigung Österreichischer
Bibliothekarinnen und Bibliothekare} 69 (2): 236--43.
\url{https://doi.org/10.31263/voebm.v69i2.1629}.

\bibitem[\citeproctext]{ref-staunton2021}
Staunton, Ciara, Carlos Andrés Barragán, Stefano Canali, Calvin Ho,
Sabina Leonelli, Matthew Mayernik, Barbara Prainsack, und Ambroise
Wonkham. 2021. {«Open Science, Data Sharing and Solidarity: Who
Benefits?»} \emph{History and Philosophy of the Life Sciences} 43 (4):
115. \url{https://doi.org/10.1007/s40656-021-00468-6}.

\bibitem[\citeproctext]{ref-musis2024}
Team MusIS. o.~J. {«Regelwerke, Thesauri, Klassifikationen, Systematiken
und Begriffslisten»}. Wiki. BSZ Wiki. Zugegriffen 5. Mai 2024.
\url{https://wiki.bsz-bw.de/display/MUSIS/Regelwerke\%2C+Thesauri\%2C+Klassifikationen\%2C+Systematiken+und+Begriffslisten}.

\bibitem[\citeproctext]{ref-community2022a}
The Turing Way Community. 2022. {«The {Turing Way}: {A} Handbook for
Reproducible, Ethical and Collaborative Research»}.
\url{https://doi.org/10.5281/ZENODO.3233853}.

\bibitem[\citeproctext]{ref-wilkinson2016}
Wilkinson, Mark D., Michel Dumontier, IJsbrand Jan Aalbersberg,
Gabrielle Appleton, Myles Axton, Arie Baak, Niklas Blomberg, u.~a. 2016.
{«The {FAIR Guiding Principles} for Scientific Data Management and
Stewardship»}. \emph{Scientific Data} 3 (1): 160018.
\url{https://doi.org/10.1038/sdata.2016.18}.

\bibitem[\citeproctext]{ref-zhang2022a}
Zhang, Lei. 2022. {«Empowering Linked Data in Cultural Heritage
Institutions: {A} Knowledge Management Perspective»}. \emph{Data and
Information Management} 6 (3): 100013.
\url{https://doi.org/10.1016/j.dim.2022.100013}.

\end{CSLReferences}

\section{Anhang}\label{anhang}

\subsection{Checkliste}\label{checkliste}

\begin{itemize}
\tightlist
\item
  \ldots{}
\end{itemize}

\subsection{Nützliche Tools}\label{nuxfctzliche-tools}

\subsubsection{Handbücher und
Leitfäden}\label{handbuxfccher-und-leitfuxe4den}

\begin{itemize}
\tightlist
\item
  Reader for Black Lives Philadephia
\end{itemize}

\subsubsection{GND-Viewers}\label{gnd-viewers}

\begin{itemize}
\tightlist
\item
  Kommentar Vorteile und Nachteile (z. B. des GND-Explorers)\\
\item
  \href{https://swb.bsz-bw.de/DB=2.104/?COOKIE=Us998,Pbszgast,I2017,B20728+,SY,NRecherche-DB,D2.104,E6b0d3db2-1,A,H,R178.197.206.37,FY}{https://swb.bsz-bw.de/}\\
\item
  GND-wiki der dnb\\
\item
  Regelwerk\\
\item
  Theoriebox
\end{itemize}

\subsubsection{Ontologien und Mappings}\label{ontologien-und-mappings}

\begin{itemize}
\tightlist
\item
  \href{https://terminology-view.lido-schema.org/vocnet/?startNode=lido00409&lang=en&uriVocItem=http://terminology.lido-schema.org/eventType}{LIDO}
\item
  Iconclass (ruft Community aktiv auf, sich antidiskriminierend zu
  beteiligen)\\
\item
  \url{https://wiki.dnb.de/display/GND/GND-Mappings+zu+externen+Thesauri}\strut \\
\item
  Link zum Opengender-Mapping (externe Publikation)
\end{itemize}

\subsection{\texorpdfstring{Metadaten aus dem Projekt
\emph{Stadt.Geschichte.Basel}}{Metadaten aus dem Projekt Stadt.Geschichte.Basel}}\label{metadaten-aus-dem-projekt-stadt.geschichte.basel}

\begin{tcolorbox}[enhanced jigsaw, colbacktitle=quarto-callout-tip-color!10!white, rightrule=.15mm, coltitle=black, left=2mm, opacitybacktitle=0.6, toptitle=1mm, title=\textcolor{quarto-callout-tip-color}{\faLightbulb}\hspace{0.5em}{Erfahrungen Stadt.Geschichte.Basel}, colback=white, colframe=quarto-callout-tip-color-frame, opacityback=0, titlerule=0mm, leftrule=.75mm, breakable, bottomtitle=1mm, bottomrule=.15mm, arc=.35mm, toprule=.15mm]

\begin{itemize}
\tightlist
\item
  Metadatenobjekte (Eltern)\\
\item
  Zugeordnete Ressourcen (Kinder)
\item
  Relation von Objekt und Media\\
\end{itemize}

\end{tcolorbox}




\end{document}
